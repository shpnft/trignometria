\pgfmathsetmacro\CosA{cos(32)}
\pgfmathsetmacro\CosB{cos(150)}
\pgfmathsetmacro\CosC{cos(243)}
\pgfmathsetmacro\CosD{cos(345)}
\pgfmathsetmacro\SomadoA{cos(345)+cos(150)}
\pgfmathsetmacro\SomadoB{(sqrt(6)+sqrt(2)-2*sqrt(3))/4}


\begin{frame}{Exercício 17}
    \begin{itemize}
        \item Usando uma calculadora, temos que
            \begin{align*}
                \cos{\SI{32}{\degree}} &= \num{\CosA} \\
                \cos{\SI{150}{\degree}} &= \num{\CosB} \\
                \cos{\SI{243}{\degree}} &= \num{\CosC} \\
                \cos{\SI{345}{\degree}} &= \num{\CosD}
            \end{align*}
        \item Ou seja, \(x=\cos{\SI{345}{\degree}}\) e \(y=\cos{\SI{150}{\degree}}\)
        \item Assim, 
            \[
                x+y=
                \begin{cases}
                    \num{\CosD} \num{\CosB} = \num{\SomadoA} \qquad &\text{resultado obtido usando calculadora}\\
                    \dfrac{\sqrt{6}+\sqrt{2}-2\sqrt{3}}{4} \approx \num{\SomadoB} \qquad &\text{resultado obtido em sala de aula}
                \end{cases}
            \]
    \end{itemize}
\end{frame}
