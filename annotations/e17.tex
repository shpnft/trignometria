\documentclass[brazilian]{article}

\usepackage{babel}
\usepackage[utf8]{inputenc}
\usepackage[T1]{fontenc}
\usepackage{lmodern}

\usepackage{amsmath}
\usepackage{siunitx}

\usepackage{tikz}
\usepackage{fp}
\usetikzlibrary{fixedpointarithmetic}
\tikzset{fixed point arithmetic}

\newcommand{\bob}[1]{\pgfmathparse{#1}\pgfmathprintnumber[use comma, precision=6]{\pgfmathresult}}

\begin{document}
\begin{itemize}
    \item Usando uma calculadora, temos que
        \begin{align*}
            \cos{\SI{32}{\degree}} &= \bob{cos(32)}\\
            \cos{\SI{150}{\degree}} &= \bob{cos(150)}\\
            \cos{\SI{243}{\degree}} &= \bob{cos(243)}\\
            \cos{\SI{345}{\degree}} &= \bob{cos(345)} 
        \end{align*}
    \item Ou seja, \(x=\cos{\SI{345}{\degree}}\) e \(y=\cos{\SI{150}{\degree}}\)
    \item Assim, 
        \[
            x+y=
            \begin{cases}
                \bob{cos(345)}-\bob{-cos(150)} = \bob{cos(345)+cos(150)} 
                \qquad &\text{(resultado obtido usando calculadora)}\\
                \dfrac{\sqrt{6}+\sqrt{2}-2\sqrt{3}}{4} \approx \bob{(sqrt(6)+sqrt(2)-2*sqrt(3))/4} 
                \qquad &\text{(resultado obtido em sala de aula)}
            \end{cases}
        \]
\end{itemize}
\end{document}
